\documentclass{article}
\usepackage[utf8]{inputenc}

\title{Advanced Information Retrieval. REMBO Algorithm}
\author{Jerome Cremers}
\date{March 2014}

\begin{document}

\maketitle

\section{Introduction}
-rembo\\
-lerot\\

In this report, we analyze our own implementation of the Random EMbedding Bayesian Optimazation (REMBO) algorithm \cite{xu2007adarank}. REMBO overcomes the issue of general Bayesian optimization approaches being restricted to problems of moderate dimensions. 


\section{Theoretical background}
- Explanation of how REMBO works
- Explanation of learning to rank models and their application
    Explanation of how lerot works

\section{Experimental design}

Evaluation: comparison between lerot which just uses all feature vectors as is

VS

using REMBO on the feature vectors first to reduce the dimensions and running lerot on that

Look out: CPU time / iterations required (depends on how iterations work, ask Masrour)

The next evaluations we could try are comparing different setups (parameters) of REMBO to each other and the baseline lerot without REMBO. For example:

k=1 vs k>1 interleaved runs comparison

d=2,4,6,etc comparison

Fineness of grid Y comparison (0.01, 0.001, 0.0001 etc)

\section{Results}

\section{Discussion}

\section{Conclusion}

\bibliographystyle{unsrt}%Used BibTeX style is unsrt
\bibliography{references}

\end{document}
